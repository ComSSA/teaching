\documentclass[a4paper,12pt]{article}
\usepackage[utf8]{inputenc}
\usepackage[T1]{fontenc}
\usepackage[margin=1in]{geometry}
\usepackage{parskip}
\usepackage{listings}
\usepackage{enumerate}
\usepackage{textcomp}
\usepackage[usenames,dvipsnames]{color}
\lstset{basicstyle=\ttfamily, basewidth=0.5em, upquote=true}

\title{ComSSA teaches UCP: test 1 revision}
\date{September 1, 2015}

\pagenumbering{arabic}

\begin{document}

\maketitle

Questions in UCP tests and exams follow a predictable format, and each
of the sections below represent a kind of question that you may be
asked in an assessment, you may have seen in a practical, or may
otherwise improve your understanding.

To better prepare you for your assessments, the questions that follow
are often slightly more difficult than you may expect to encounter, but
they will help hone your skills and ensure that you pay extra attention
when answering the real questions.

Dave likes to write tricky questions with subtle pitfalls, many of
which test your understanding of \emph{entirely valid code} that will
compile but either run incorrectly, run surprisingly, or be otherwise
less than ideal. Watch out for these warning signs:

\begin{itemize}
	\item Preprocessor macros defined
	\begin{itemize}
		\item without parentheses around each use of a parameter
		\item without parentheses around the resultant expression
		\item with a semicolon at the end, when there shouldn't be
	\end{itemize}
	\item Loops and other control structures in which
	\begin{itemize}
		\item counters or other key variables are modified
		\item pointers are modified while they're being used
	\end{itemize}
	\item Unusual uses of control structures such as
	\begin{itemize}
		\item \texttt{case} statements without
		      \texttt{break} statements
		\item \texttt{for} and \texttt{while} loops with
		      \texttt{break} and \texttt{continue} statements
	\end{itemize}
	\item Variables that are modified once and never used again
\end{itemize}

The following are some other hints that may be useful for today:

\begin{itemize}
	\item A semicolon on its own is called a ``null statement'' and
	      it does nothing at all
	\item \texttt{x = y} is an expression that ``returns''
	      the new value of \texttt{x}
	\item \texttt{v = w = x = y;} is thus equivalent to
	      \texttt{x = y; w = x; v = w;}
\end{itemize}

Questions marked * are probably beyond the scope of the assessment.

\newpage

\section{Terminology}

Explain the meaning of the following:

\begin{enumerate}[a)]
	\item \texttt{[17065012@saeshell01p \textasciitilde]\$ ./a \&}
	\item \lstinline@alias gcc='gcc -Wall -ansi -pedantic'@
	\item \lstinline@#define E 2.71828183f@
	\item \lstinline@#define TRIANGLE(b, h) ((b) * (h) / 2)@
	\item \lstinline@#include <header.h>@
	\item \lstinline@#include "header.h"@
	\item \lstinline@#ifdef DEBUG@\\
	      \lstinline@        #define debug(s) printf("%s\n", s)@\\
	      \lstinline@#else@\\
	      \lstinline@        #define debug(s)@\\
	      \lstinline@#endif@
	\item \lstinline@#ifndef HEADER_H@\\
	      \lstinline@#define HEADER_H@\\
	      \lstinline@...@\\
	      \lstinline@#endif@
	\item \lstinline@int *foo, bar;@
	\item \lstinline@double foo, bar = 42.0;@
	\item \lstinline@unsigned char foo;@
	\item[* l)] \lstinline@extern int foo;@
	\item[* m)] \lstinline@extern int foo();@
	\item[n)] \lstinline@int foo(void);@
	\item[o)] \lstinline@int foo(int, int, int);@
	\item[p)] \lstinline@int foo(int, int y, int z);@
	\item[q)] \lstinline@static int foo(int x) { return x * x; }@
	\item[r)] \lstinline@int **foo(void);@
	\item[s)] \lstinline@int *(*foo)(void);@
	\item[t)] \lstinline@int (**foo)(void);@
	\item[u)] \lstinline@typedef int *(*foo)(void);@
	\item[* v)]
	      \lstinline@typedef int *(*foo(float (*)(float)))(void);@
	\item[w)] \lstinline@int **foo = (int **) malloc(sizeof(int *));@
	\item[x)] \lstinline@int *foo; ... if (!foo) ...@
	\item[y)] \lstinline@int *foo; ... if (!*foo) ...@
\end{enumerate}

\newpage

\section{Discussion}

\begin{enumerate}[a)]
	\item Why should you put forward declarations in header files?
	\item Which of the following should reside in a header file and
	      why?
	\begin{enumerate}[i)]
		\item[* a)] Global variable declaration
		      (not \texttt{extern})
		\item[* b)] Global variable declaration
		      (\texttt{extern})
		\item[c)] Function declaration (not \texttt{static})
		\item[d)] Function declaration (\texttt{static})
		\item[e)] Function definition (not \texttt{static})
		\item[f)] Function definition (\texttt{static})
		\item[g)] \texttt{typedef} declaration
		\item[h)] \texttt{\#include} directive
		\item[i)] \texttt{\#include} guard
	\end{enumerate}
	\item What is the return type of \texttt{main()} and why?
	\item What is the return type of \texttt{malloc()} and why?
	\item What is the parameter type of \texttt{free()} and why?
	\item What is an L-value?
	\item What kinds of memory errors can \texttt{valgrind} catch?
	\item Why is it a good idea to set a pointer to \texttt{NULL}
	      after freeing it?
\end{enumerate}

\newpage

\section{Pointer diagrams}

Say you have the following declarations:

\begin{lstlisting}
int x = 13;
int y = 42;
int *p = NULL;
int *q = NULL;
int **s = NULL;
int **t = NULL;
\end{lstlisting}

For each of the following code snippets, draw a diagram showing all
pointer relationships created, and state the resulting values of
\texttt{x} and \texttt{y}. Assume that the declarations ``reset''
between questions --- in other words, the questions are independent of
one another.

a)

% t -> q -> y
% p -> x
% (x, y) = (42, 13)
% y += y - x;
% (x, y) = (42, -16)

\begin{lstlisting}
int z;
t = &q;
q = &y;
p = &x;
z = x;
x = y;
y = z;
*q += **t - *p;
\end{lstlisting}

* b)

% (p) -> malloc [ ? ]
% (q) -> malloc [ 42 ]
% (s, t) -> malloc [ q ]
% x = 42 + 42;
% (x, y) = (84, 42)

\begin{lstlisting}
p = (int *) malloc(sizeof(int));
q = (int *) malloc(sizeof(int));
s = t = (int **) malloc(sizeof(int *));
*s = p;
*t = q;
**s = x;
**t = y;
x = **s + **t;
\end{lstlisting}

* c)

% 13 / 4 == 3
% 42 / 4 == 10
% i == 0 < 10 (TRUE)
% i == 1 < 3 (TRUE)
% i == 2 < 10 (TRUE)
% i == 3 < 3 (FALSE)
% swap (s, t) three times
% s -> p -> x
% t -> q -> y
% x *= 2;
% (x, y) == (26, 42)

\begin{lstlisting}
int **h, i;
p = &x;
q = &y;
s = &q;
t = &p;
for (i = 0; i < **s / 4; i++) {
	h = s;
	s = t;
	t = h;
}
**s *= 2;
\end{lstlisting}

\newpage

\section{Outputs of C programs}

For each of the following C programs, determine what will be sent to
\texttt{stdout}.

a) % Dave's classic macro trick question: even

\begin{lstlisting}
#include <stdio.h>
#define ODD(number) (number % 2 == 1)

int main(void) {
	if (ODD(2 + 2))
		printf("odd\n");
	else
		printf("even\n");
	return 0;
}
\end{lstlisting}

b) % 2\n4\n8\n...512\n1024\n

\begin{lstlisting}
#include <stdio.h>

void foo(void *p) {
	* (int *) p *= 2;
	printf("%d\n", * (int *) p);
}

int main(void) {
	int i = 1, j = 1;
	do {
		i++;
		foo((void *) &j);
	} while (i <= 10);
	return 0;
}
\end{lstlisting}

* c) % Recursive factorial function: 120

\begin{lstlisting}
#include <stdio.h>

double foo(unsigned int x) {
	if (x < 2)
		return 1;
	else
		return foo(x - 1) * (double) x;
}

int main(void) {
	printf("%.0f\n", foo(5));
	return 0;
}
\end{lstlisting}

\newpage

\section{Error inspection}

Each of the following code snippets contains an error. Explain what is
wrong. (You don't need to show how to fix it, but you can if it will
assist your explanation.)

a) % Missing forward declaration for square()

\begin{lstlisting}
double cube(double x) {
	return x * square(x);
}

double square(double x) {
	return x * x;
}
\end{lstlisting}

b) % Integer swapping function

\begin{lstlisting}
void swap(int a, int b) {
	int t;
	t = a;
	a = b;
	b = t;
}
\end{lstlisting}

c) % Extraneous semicolon in the macro replacement

\begin{lstlisting}
#include <stdio.h>
#define PRINT_NUMBER(x) printf("%.17f", x);

void foo(int condition) {
	if (condition)
		PRINT_NUMBER(1.0 / 3.0);
	else
		PRINT_NUMBER(1.0 / 7.0);
}
\end{lstlisting}

\newpage

\section{Effects of C functions}

Describe the effect of the following functions, using examples as
needed:

a)

\begin{lstlisting}
typedef void (*callback)(void *);
void fun0(void *);
void fun1(void *);
void fun2(void *);
void fun3(void *);

callback foo(void) {
	int bar;
	scanf("%d", &bar);
	switch (bar) {
		case 1:
			return &fun1;
		case 2:
			return &fun2;
		case 3:
			return &fun3;
		default:
			return &fun0;
	}
}
\end{lstlisting}

b)

\begin{lstlisting}
int foo(void) {
	char bar;
	int baz = 0;
	do {
		scanf("%c", &bar);
		if (bar >= '0' && bar <= '9')
			baz = (baz * 16) + (bar - '0');
		if (bar >= 'A' && bar <= 'F')
			baz = (baz * 16) + 10 + (bar - 'A');
		if (bar >= 'a' && bar <= 'f')
			baz = (baz * 16) + 10 + (bar - 'a');
	} while (
		(bar >= '0' && bar <= '9') ||
		(bar >= 'A' && bar <= 'F') ||
		(bar >= 'a' && bar <= 'f')
	);
	return baz;
}
\end{lstlisting}

\newpage

\section{Makefiles and dependencies}

Consider a chat client and server made up of the following C files:

\begin{minipage}{.3\textwidth}
\begin{lstlisting}[caption=client.c,frame=tlrb]{Name}
#include "client.h"
#include "network.h"

...
\end{lstlisting}
\end{minipage}\hfill
\begin{minipage}{.3\textwidth}
\begin{lstlisting}[caption=server.c,frame=tlrb]{Name}
#include "server.h"
#include "network.h"

...
\end{lstlisting}
\end{minipage}\hfill
\begin{minipage}{.3\textwidth}
\begin{lstlisting}[caption=network.c,frame=tlrb]{Name}
#include "network.h"
#define PORT 6697

...
\end{lstlisting}
\end{minipage}

\begin{minipage}{.3\textwidth}
\begin{lstlisting}[caption=client.h,frame=tlrb]{Name}
#include "interface.h"

...
\end{lstlisting}
\end{minipage}\hfill
\begin{minipage}{.3\textwidth}
\begin{lstlisting}[caption=server.h,frame=tlrb]{Name}
#include <event.h>

...
\end{lstlisting}
\end{minipage}\hfill
\begin{minipage}{.3\textwidth}
\begin{lstlisting}[caption=network.h,frame=tlrb]{Name}
#include <socket.h>

...
\end{lstlisting}
\end{minipage}

\begin{minipage}{.3\textwidth}
\begin{lstlisting}[caption=interface.c,frame=tlrb]{Name}
#include "interface.h"

...
\end{lstlisting}
\end{minipage}\hfill
\begin{minipage}{.3\textwidth}
\begin{lstlisting}[caption=interface.h,frame=tlrb]{Name}
#include <curses.h>

...
\end{lstlisting}
\end{minipage}\hfill
\begin{minipage}{.3\textwidth}
\begin{lstlisting}[caption=common.h,frame=tlrb]{Name}
#include <stdlib.h>

#define VERSION 4.2.0
\end{lstlisting}
\end{minipage}

\begin{enumerate}[a)]
	\item What files would serve as targets in a makefile, and why?
	\item What are the dependencies for each target, and why?
	\item If the file \texttt{interface.h} changes, which files
	      would need to be recompiled, and how does make figure
	      this out?
	\item Produce a suitable makefile for this project. Make good
	      use of makefile variables, and include a suitable
	      “clean” rule. The executable files should be named
	      \texttt{client} and \texttt{server}.
\end{enumerate}

\newpage

\section{Export pointers and function pointers}

You’ve been asked to help write code for a simple video game with an
overhead two-dimensional display. The game consists of a grid of
positions $(x, y)$ where $1 \leq x \leq 80$ and $1 \leq y \leq 20$.
The player’s character may reside in any of the 1600 $(x, y)$ pairs,
except where there is an object in the way, and the character must
“wrap around” to an opposite side when the player tries to move beyond
the boundaries provided.

Your task is to write a function called \texttt{keyboard} that is
called whenever the player presses a key to move their character. The
key will be given to you in the form of a \texttt{char}, and will be
\texttt{\textquotesingle{}h\textquotesingle},
\texttt{\textquotesingle{}j\textquotesingle},
\texttt{\textquotesingle{}k\textquotesingle}, or
\texttt{\textquotesingle{}l\textquotesingle} to move left, down, up, or
right respectively. Except for wrapping around an edge, either $x$ or
$y$ should change by $\pm 1$ when a player successfully moves in any
direction.

You may assume that the following constants have already been written:

\begin{lstlisting}
#define WIDTH 80
#define HEIGHT 20
\end{lstlisting}

Your function must match the following prototype:

\begin{lstlisting}
void keyboard(
	char key,
	void (*location)(int *x, int *y),
	int (*occupied)(int x, int y),
	void (*move)(int x, int y)
);
\end{lstlisting}

You may assume that the three function pointer parameters will be
supplied with valid pointers to functions that satisfy the behaviours
below.

\texttt{location} exports the player character’s current location using
the pointers \texttt{x} and \texttt{y} supplied by the caller.
\texttt{occupied} returns \texttt{1} if the space at $(x, y)$ is
occupied, and \texttt{0} if the space is not occupied. \texttt{move}
unconditionally moves the player’s character to the space at $(x, y)$,
without checking or validating anything about the location given.

If the player presses any key that wasn’t mentioned above, your
function should do nothing. Otherwise, your function should first check
to see whether or not the player is trying to move beyond an edge.
Based on that, your function should then check to see whether the
target location is occupied. If not, your function should move the
player’s character to the target location.

\newpage

\section{Memory usage*}

For each of the following code snippets, state how much additional
memory would be used at runtime if you were to add them to a program.
You may assume that none of the functions are being called ``right
now'', so the stack is not affected, and that:

\begin{itemize}
	\item the size of any \texttt{int} is 4 bytes
	\item the size of any pointer is 8 bytes
	\item the size of any function is 30 bytes
\end{itemize}

a) % 0 + 0 + 30 = 30 bytes

\begin{lstlisting}
extern int errno;
char *strerror(int errnum);

int foo(int bar) {
	if (some_syscall(bar) == -1) {
		fprintf(stderr, "%s\n", strerror(errno));
		return errno;
	}
	return 0;
}
\end{lstlisting}

b) % 0 + 8 + 0 + 30 + 8 + 0 = 46 bytes

\begin{lstlisting}
extern int foo;
int *bar;
int *baz(void);
int *qux(void) {
	...
}
int (*bat)(void);
typedef int (*cat)(void);
\end{lstlisting}

\end{document}
