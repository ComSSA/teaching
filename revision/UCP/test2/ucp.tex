\documentclass[a4paper,12pt]{article}
\usepackage[utf8]{inputenc}
\usepackage[T1]{fontenc}
\usepackage[margin=1in]{geometry}
\usepackage{parskip}
\usepackage{listings}
\usepackage{enumerate}
\usepackage{textcomp}
\usepackage{textgreek}
\usepackage[usenames,dvipsnames]{color}
\lstset{basicstyle=\ttfamily, basewidth=0.5em, upquote=true}

\title{ComSSA teaches UCP: test 2 revision}
\date{October 10, 2015}

\pagenumbering{arabic}

\begin{document}

\maketitle

Questions in UCP tests and exams follow a predictable format, and each
of the sections below represent a kind of question that you may be
asked in an assessment, you may have seen in a practical, or may
otherwise improve your understanding.

To better prepare you for your assessments, the questions that follow
are often slightly more difficult than you may expect to encounter, but
they will help hone your skills and ensure that you pay extra attention
when answering the real questions.

Dave likes to write tricky questions with subtle pitfalls, many of
which test your understanding of \emph{entirely valid code} that will
compile but either run incorrectly, run surprisingly, or be otherwise
less than ideal. Watch out for these warning signs:

\begin{itemize}
	\item When Dave mentions a ``pointer to an array'', he nearly
	      always actually means a ``pointer to the first element of
	      an array'', but pointers to entire arrays are also a
	      real concept, even if they are used rarely in practice.
\end{itemize}

The following are some other hints that may be useful for today:

\begin{itemize}
	\item When a function is referred to in a manner such as
	      \lstinline@function(3)@, you can learn more about the
	      function in section 3 of the Unix manual by typing
	      \lstinline@man 3 function@.
	\item The exact data type of an expression is what one would
	      write to declare a variable for it, without the name of
	      the variable itself; for example, the type of a pointer
	      to a function with no formal parameters that returns an
	      \lstinline@int@ is \lstinline@int (*)(void)@.
	\item The expressions \lstinline@foo[i]@ and
	      \lstinline@*(foo + i)@ are equivalent, and either form
	      can be used to access both fixed and
	      \lstinline@malloc(3)@'d arrays equally well.
	\item Whenever the name of an array is used in any expression,
	      the array is temporarily converted to a pointer to its
	      first element in a process that is colloquially known as
	      ``array decay'', except for the expressions
	      \lstinline@sizeof(array)@ and \lstinline@&array@.
	\item The only two places where \lstinline@[]@ makes sense are
	      in the first dimension of an array declaration, where the
	      compiler fills in the size based on the initialiser, and
	      in the first dimension of a function parameter, where it
	      is syntactic sugar for \lstinline@*@.
\end{itemize}

Questions marked * are probably beyond the scope of the assessment.

\newpage

\section{Declarations, expressions, and data types}

Explain the meanings and the data types of the variables and/or
expressions below:

\begin{enumerate}[a)]
\item\lstinline@argc; argv; argv[0]; argv[4];@
\item\lstinline@int *foo = malloc(5 * sizeof int); foo; &foo; *foo;@
\item\lstinline@int *foo = malloc(5 * sizeof int); foo[3]; &foo[3];@
\item\lstinline@int *foo = malloc(5 * sizeof int); foo + 3; *(foo + 3);@
\item\lstinline@int foo[5]; foo; &foo; *foo;@
\item\lstinline@int foo[5]; foo[3]; &foo[3];@
\item\lstinline@int foo[5]; foo + 3; *(foo + 3);@
\item\lstinline@int foo[3][2] = { { 4, 5 }, { 6, 7 }, { 8, 9 } };@
\item\lstinline@int foo[5][4]; foo; &foo; *foo;@
\item\lstinline@int foo[5][4]; foo[3]; &foo[3];@
\item\lstinline@int foo[5][4]; foo[3][1]; &foo[3][1];@
\item\lstinline@int foo[5][4]; foo + 3; *(foo + 3);@
\item\lstinline@int foo[5][4]; *(foo + 3) + 1; *(*(foo + 3) + 1);@
\item\lstinline@int *foo[5][4], **bar[5][4];@
\item\lstinline@int (*foo)[5][4], (*bar(void))[5][4];@
\item[* p)]\lstinline@int *(*foo[5][4])(int bar, int);@
\item[q)]\lstinline@int foo(float bar[], size_t length);@
\item[r)]\lstinline@int foo(float bar[][COLS], size_t rows);@
\item[s)]\lstinline@int foo(float bar[ROWS][COLS]);@
\item[t)]\lstinline@char *foo = "Hello, world!";@
\item[u)]\lstinline@char foo[] = "Hello, world!";@
\item[* v)]\lstinline@struct Point { int x, y; };@
\item[* w)]\lstinline@struct Point { int x, y; } place;@
\item[* x)]\lstinline@struct { int x, y; } place = { 13, 420 };@
\item[y)]\lstinline@typedef struct { int x, y; } Point;@
\item[z)]\lstinline@typedef struct Point { int x, y; } Point;@
\item[\textalpha)]\lstinline@typedef struct Foo { char *bar; } Foo;@
\item[\textbeta)]\lstinline@typedef struct Foo { char bar[9]; } Foo;@
\item[\textgamma)]\lstinline@foo.a; foo.b->c.d; foo.e[7]->f[6].g;@
\item[\textdelta)]\lstinline@foo->a; foo->b.c->d; foo->e + 13;@
\end{enumerate}

\newpage

\section{Evaluating the outputs of \lstinline@printf(3)@ calls}

For each of the questions below, write what would be sent to
\lstinline@stdout@:

\begin{enumerate}[a)]
	\item\lstinline@printf("~%d~", 13);@
	\item\lstinline@printf("~%+d~", 13);@
	\item\lstinline@printf("~%5d~", 13);@
	\item\lstinline@printf("~%+5d~", 13);@
	\item\lstinline@printf("~%-5d~", 13);@
	\item\lstinline@printf("~%+-5d~", 13);@
	\item\lstinline@printf("~%05d~", 13);@
	\item\lstinline@printf("~%+05d~", 13);@
	\item\lstinline@printf("~%3.9f~", 13.0);@
	\item\lstinline@printf("~%+3.9f~", 13.0);@
	\item\lstinline@printf("~%9.3f~", 13.0);@
	\item\lstinline@printf("~%+9.3f~", 13.0);@
	\item\lstinline@printf("~%-9.3f~", 13.0);@
	\item\lstinline@printf("~%+-9.3f~", 13.0);@
	\item\lstinline@printf("~%09.3f~", 13.0);@
	\item\lstinline@printf("~%+09.3f~", 13.0);@
	\item\lstinline@printf("%s,%6s!", "Hello", "world");@
	\item\lstinline@printf("~%d~", '7' - '0');@
	\item\lstinline@printf("~%c~", 7 + '0');@
\end{enumerate}

\newpage

\section{Discussion}

\begin{enumerate}[a)]
	\item What values can \texttt{fopen(3)} return and why?
	\item What values can \texttt{fgetc(3)} return and why?
	\item What values can \texttt{scanf(3)} return and why?
	\item What values can \texttt{fgets(3)} return and why?
	\item Why is it incorrect to read a file by looping until
	      \texttt{feof(3)} returns true?
	\item How can one distinguish reaching the end of a file from
	      an I/O error?
	\item How should a C program handle an I/O error?
\end{enumerate}

\newpage

\section{\lstinline@struct@ declaration and allocation}

For each of the following descriptions:

\begin{enumerate}[i)]
	\item Write a suitable type declaration (or set of type
	      declarations) for representing the data.
	\item Show how to dynamically allocate the necessary memory,
	      and initialise any obvious fields.
\end{enumerate}

Assume that all strings have an upper limit of 168 characters.

(Note: based on your declarations, it should be possible to use and
manipulate all the information through a single pointer variable.)

\begin{enumerate}[a)]
	\item A pharmaceutical drug, with a generic name, a trade name,
	      a year of discovery, a year of approval, a molecular mass
	      in g/mol, and a half-life in hours.
	\item A digital audio file, with an artist name, album name,
	      track name, sampling rate in Hz, and an array of audio
	      samples, where each sample is an integer. When allocating
	      and initialising the structure, assume that the integer
	      variable \lstinline@samples@ exists and contains the
	      number of samples in the digital audio file.
	\item A web address, or URL, with a host name, a port number,
	      a path, an array of fields, and a fragment name. Each
	      field consists of a name and a value, both of which are
	      strings, as well as the path. When allocating and
	      initialising the structure, assume that the integer
	      variable \lstinline@fields@ exists and contains the
	      number of fields.
	\item A text file, consisting of a file name, as well as a
	      double ended, doubly linked list of lines of text. When
	      allocating and initialising the structure, assume that
	      the text file contains no lines. Writing any code to
	      manipulate the linked list is not necessary.
\end{enumerate}

\newpage

\section{Pointer diagrams}

Say you have the following declarations:

\begin{lstlisting}
int x = 13;
int y[] = { 1, 4, 9 };
int z[][2] = { 2, 4, 6, 8 };
int *p = NULL;
int *q[2] = { &z[0][2], &y[1] };
int **s;
int **t[2];
\end{lstlisting}

For each of the following code snippets, draw a diagram showing all
pointer relationships created, and state the resulting value of
\texttt{x}. Assume that the declarations ``reset'' between questions
--- in other words, the questions are independent of one another.

a)

% s -> p -> x
% t[0] -> q[1] -> y[1]
% t[1] -> q[0] -> z[0][2]
% x = 4 + 6;

\begin{lstlisting}
s = &p;
p = &x;
t[0] = &q[1];
t[1] = &q[0];
s[0][0] = **t[0] + **t[1];
\end{lstlisting}

% t[0] -> p
% t[1] -> q[0]
% p -> malloc [int, int]
% p[0] = y[2];
% p[1] = &z[0][2] - &z[0][0];
% (p[0], p[1]) == (9, 2)
% q[0] -> x
% x = 2 * 9;

b)

\begin{lstlisting}
*t = &p;
t[z[1][0] - 5] = q;
p = (int *) malloc(2 * sizeof(int));
*(p + 0) = y[sizeof(z) / sizeof(z[0])];
*(p + 1) = q[0] - *z;
q[0] = &x;
t[1][0][0] = t[0][0][1] * ***t;
\end{lstlisting}

\end{document}
